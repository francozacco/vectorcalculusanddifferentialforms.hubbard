\documentclass[11pt]{article}
\usepackage{amssymb}
\usepackage{amsthm}
\usepackage{enumitem}
\usepackage{amsmath}
\usepackage{bm}
\usepackage{adjustbox}
\usepackage{mathrsfs}
\usepackage{graphicx}
\usepackage{siunitx}
\usepackage[mathscr]{euscript}

\title{\textbf{Solved selected problems of Calculus on Manifolds by Michael Spivak}}
\author{Franco Zacco}
\date{}

\addtolength{\topmargin}{-3cm}
\addtolength{\textheight}{3cm}

\newcommand{\R}{\mathbb{R}}
\newcommand{\Q}{\mathbb{Q}}
\newcommand{\C}{\mathbb{C}}
\newcommand{\N}{\mathbb{N}}
\newcommand{\hatr}{\bm{\hat{r}}}
\newcommand{\hatx}{\bm{\hat{x}}}
\newcommand{\haty}{\bm{\hat{y}}}
\newcommand{\hatz}{\bm{\hat{z}}}
\newcommand{\hatth}{\bm{\hat{\theta}}}
\newcommand{\hatphi}{\bm{\hat{\phi}}}
\newcommand{\hatrho}{\bm{\hat{\rho}}}
\newcommand{\ei}[1]{\vec{\bm{e}}_#1}
\newcommand{\bvec}[1]{\vec{\bm{#1}}}
\newcommand{\tr}{\text{tr}}
\theoremstyle{definition}
% \newtheorem*{solution*}{Solution}
% \renewcommand*{\proofname}{Solution}

\begin{document}
\maketitle
\thispagestyle{empty}

\section*{2 - Differentiation}
\subsection*{Basic Definitions}
\begin{proof}{\textbf{2-2.}}
Let $f:\R^2 \to \R$ be independent of the second variable i.e. for each
$x \in \R$ we have that $f(x,y_1) = f(x, y_2)$ for all $y_1, y_2 \in \R$.
\\
Then we can take some $y_0 \in \R$ and define a function $g:\R \to \R$ such
that $g(x) = f(x,y_0)$, then, for any $y\in \R$ we get that
$f(x,y) = f(x,y_0) = g(x)$ because $f$ is independent of the second variable.
\\\\
On the other hand, suppose we have a function $g:\R \to \R$ such that
$g(x) = f(x,y)$ then since this is true for any $y\in \R$ then we can write 
that
$$f(x,y_1) = g(x) = f(x, y_2)$$
for any $y_1, y_2 \in \R$, which implies that $f$ is independent of the second
variable.
\\\\
The Jacobian of $f$ at $(a,b)$ i.e. $f'(a,b)$ is given by
\begin{align*}
    f'(a,b) = (D_1f(a, b),~D_2f(a, b))
\end{align*}
Since $g(x) = f(x,y)$ then $D_1f(a,b) = D_1g(a) = g'(a)$ therefore
\begin{align*}
    f'(a,b) = (g'(a),~0)
\end{align*}
\end{proof}

\begin{proof}{\textbf{2-3.}}
A function $f:\R^2 \to \R$ is independent of the first variable if for each
$y\in \R$ we have that$f(x_1, y) = f(x_2, y)$ for all $x_1, x_2 \in \R$.
\\\\
The Jacobian $f'(a,b)$ in this case is
\begin{align*}
    f'(a,b) = (D_1f(a,b),~D_2f(a,b)) = (0,~D_2f(a,b))
\end{align*}
In the same way as we did in problem 2-2, we can show that $f$ is independent
of the first variable if and only if there is a function $g:\R \to \R$
such that $g(y) = f(x,y)$.
\\
So we get that $D_2f(a,b) = D_2g(b) = g'(b)$ and therefore
\begin{align*}
    f'(a,b) = (0, g'(b))
\end{align*}
Finally, the functions that are independent of the first and the second variable
are the functions of the form $f(x, y) = c$ where $c$ is a constant.
\end{proof}

\cleardoublepage
\begin{proof}{\textbf{2-4.}}
Let us define $f:\R^2\to \R$ by
\begin{align*}
    f(x) = \begin{cases}
        |x|\cdot g\big(\frac{x}{|x|}\big) & x\neq 0\\
        0  & x = 0
    \end{cases}
\end{align*}
Where $g$ is a continuous real-valued function on the unit circle\\
$\{x \in \R^2~:~|x| = 1\}$ such that $g(0,1) = g(1,0) = 0$ and $g(-x) = -g(x)$.
\begin{itemize}
\item [(a)] If $x\in\R^2$ and $h:\R \to \R$ defined by $h(t) = f(tx)$, we want
to show that $h$ is differentiable.
\\
Suppose $x \neq 0$, also, let $a \in \R$ and let
$\lambda(k) = k|x|g(\frac{x}{|x|})$ then
\begin{align*}
    &\lim_{k \to 0} \frac{|h(a + k) - h(a) - \lambda(k)|}{|k|}
    = \lim_{k \to 0} \frac{|f((a + k)x) - f(ax) - \lambda(k)|}{|k|} =\\
    &\qquad= \lim_{k \to 0}
    \frac{1}{|k|}
    \bigg||(a + k)x|g\bigg(\frac{(a + k)x}{|(a + k)x|}\bigg)
    - |ax|g\bigg(\frac{ax}{|ax|}\bigg)
    - k|x|g\bigg(\frac{x}{|x|}\bigg)\bigg|
\end{align*}
Suppose $a + k > 0$, and suppose $a > 0$ then
\begin{align*}
    &\frac{1}{|k|}
    \bigg||(a + k)x|g\bigg(\frac{(a + k)x}{|(a + k)x|}\bigg)
    - |ax|g\bigg(\frac{ax}{|ax|}\bigg)
    - k|x|g\bigg(\frac{x}{|x|}\bigg)
    \bigg| = \\
    &= \frac{1}{|k|}
    \bigg|(a + k)|x|g\bigg(\frac{x}{|x|}\bigg)
    - a|x|g\bigg(\frac{x}{|x|}\bigg)
    - k|x|g\bigg(\frac{x}{|x|}\bigg)
    \bigg|\\
    &= 0
\end{align*}
If $a < 0$ then
\begin{align*}
    |ax|g\bigg(\frac{ax}{|ax|}\bigg)
    = -a|x|g\bigg(-\frac{ax}{a|x|}\bigg)
    = a|x|g\bigg(\frac{x}{|x|}\bigg)
\end{align*}
Where we used that $g(-x) = -g(x)$ and hence
\begin{align*}
    &\frac{1}{|k|}
    \bigg||(a + k)x|g\bigg(\frac{(a + k)x}{|(a + k)x|}\bigg)
    - |ax|g\bigg(\frac{ax}{|ax|}\bigg)
    - k|x|g\bigg(\frac{x}{|x|}\bigg)
    \bigg| =\\
    &= \frac{1}{|k|}
    \bigg|(a + k)|x|g\bigg(\frac{x}{|x|}\bigg)
    - a|x|g\bigg(\frac{x}{|x|}\bigg)
    - k|x|g\bigg(\frac{x}{|x|}\bigg)
    \bigg|\\
    &= 0
\end{align*}
Now suppose $a + k < 0$ then
\begin{align*}
    &\frac{1}{|k|}
    \bigg||(a + k)x|g\bigg(\frac{(a + k)x}{|(a + k)x|}\bigg)
    - |ax|g\bigg(\frac{ax}{|ax|}\bigg)
    - k|x|g\bigg(\frac{x}{|x|}\bigg)
    \bigg| = \\
    &= \bigg|-(a + k)|x|g\bigg(-\frac{(a + k)x}{(a + k)|x|}\bigg)
    - a|x|g\bigg(\frac{x}{|x|}\bigg)
    - k|x|g\bigg(\frac{x}{|x|}\bigg)
    \bigg|\\
    &= \bigg|(a + k)|x|g\bigg(\frac{x}{|x|}\bigg)
    - a|x|g\bigg(\frac{x}{|x|}\bigg)
    - k|x|g\bigg(\frac{x}{|x|}\bigg)
    \bigg|\\
    &= 0
\end{align*}
Finally if $a + k = 0$
\begin{align*}
    &\frac{1}{|k|}
    \bigg||(a + k)x|g\bigg(\frac{(a + k)x}{|(a + k)x|}\bigg)
    - |ax|g\bigg(\frac{ax}{|ax|}\bigg)
    - k|x|g\bigg(\frac{x}{|x|}\bigg)
    \bigg| = \\
    &=\frac{1}{|k|}
    \bigg| k|x|g\bigg(\frac{ax}{|ax|}\bigg)
    - k|x|g\bigg(\frac{x}{|x|}\bigg)
    \bigg|\\
    &= 0
\end{align*}
In the case $x = 0$ we see that $\lambda(k) = 0$ and hence
\begin{align*}
    &\frac{1}{|k|}
    \bigg||(a + k)x|g\bigg(\frac{(a + k)x}{|(a + k)x|}\bigg)
    - |ax|g\bigg(\frac{ax}{|ax|}\bigg)
    - k|x|g\bigg(\frac{x}{|x|}\bigg)
    \bigg| = \\
    &=\frac{1}{|k|} \bigg| 0 - 0 - 0 \bigg|\\
    &= 0
\end{align*}
Therefore $h$ is differentiable.
\item [(b)] Let us compute $Df(0,0)$ as follows
\begin{align*}
    &\lim_{(h,k) \to (0,0)} \frac{|f(0 + (h,k)) - f(0) - Df(0,0)(h,k)|}{|(h,k)|}\\
    &= \lim_{(h,k) \to (0,0)}
    \frac{1}{|(h,k)|}\bigg|
    |(h,k)|g\bigg(\frac{(h,k)}{|(h,k)|}\bigg)  - Df(0,0)(h,k)\bigg|
\end{align*}
Let $h = 0$ then
\begin{align*}
    &\frac{1}{|(0,k)|}\bigg|
    |(0,k)|g\bigg(\frac{(0,k)}{|(0,k)|}\bigg)  - Df(0,0)(0,k)\bigg|\\
    &=\frac{1}{|k|}\bigg|
    |k|g\bigg(0,\frac{k}{|k|}\bigg) - D_2f(0,0)k\bigg|\\
    &\leq \frac{1}{|k|}\bigg|
    |k|g\bigg(0,\frac{k}{|k|}\bigg)\bigg|
    + \frac{1}{|k|}\bigg|D_2f(0,0)k\bigg|\\
    &= \bigg|g\bigg(0,\frac{k}{|k|}\bigg)\bigg|
    + \bigg|D_2f(0,0)\bigg|
\end{align*}
We see that $g(0, k/|k|) = g(0, 1) = 0$ if $k > 0$, if $k < 0$ then
$g(0, k/|k|) = g(0, -1) = g(-(0, 1)) = -g(0,1) = 0$, then we get that
\begin{align*}
    0 \leq \frac{1}{|(0,k)|}\bigg|
    |(0,k)|g\bigg(\frac{(0,k)}{|(0,k)|}\bigg)  - Df(0,0)(0,k)\bigg|
    \leq \bigg|D_2f(0,0)\bigg|
\end{align*}
In the same way, if we let $k = 0$ instead
\begin{align*}
    &\frac{1}{|(h,0)|}\bigg|
    |(h,0)|g\bigg(\frac{(h,0)}{|(h,0)|}\bigg)  - Df(0,0)(h,0)\bigg|\\
    &=\frac{1}{|h|}\bigg|
    |h|g\bigg(\frac{h}{|h|},0\bigg) - D_1f(0,0)h\bigg|\\
    &\leq \frac{1}{|h|}\bigg|
    |h|g\bigg(\frac{h}{|h|},0\bigg)\bigg|
    + \frac{1}{|h|}\bigg|D_1f(0,0)h\bigg|\\
    &= \bigg|g\bigg(\frac{h}{|h|}, 0\bigg)\bigg|
    + \bigg|D_1f(0,0)\bigg|
\end{align*}
Then since $g(0,1) = 0$ we get that
\begin{align*}
    0 \leq \frac{1}{|(h,0)|}\bigg|
    |(h,0)|g\bigg(\frac{(h,0)}{|(h,0)|}\bigg)  - Df(0,0)(h,0)\bigg|
    \leq \bigg|D_1f(0,0)\bigg|
\end{align*}
Therefore must be that $Df(0,0) = 0$ for the limit 
\begin{align*}
    \lim_{(h,k) \to (0,0)} \frac{|f(0 + (h,k)) - f(0) - Df(0,0)(h,k)|}{|(h,k)|}
\end{align*}
to converge to 0.
\\
Then for any path taken by $h$ and $k$ the limit becomes
\begin{align*}
    &\lim_{(h,k) \to (0,0)} \frac{|f(0 + (h,k)) - f(0) - Df(0,0)(h,k)|}{|(h,k)|}\\
    &=\lim_{(h,k) \to (0,0)}
    \frac{1}{|(h,k)|}\bigg|
    |(h,k)|g\bigg(\frac{(h,k)}{|(h,k)|}\bigg)  - 0\bigg|\\
    &=\lim_{(h,k) \to (0,0)}
    \bigg|g\bigg(\frac{(h,k)}{|(h,k)|}\bigg)\bigg|
\end{align*}
But $g$ is only defined at the unit circle so this limit can only be 0 if
$g = 0$ and hence $f$ is not continuous at $(0,0)$.
\end{itemize}
\end{proof}

\cleardoublepage
\begin{proof}{\textbf{2-5.}}
Let $f:\R^2 \to \R$ be defined by
\begin{align*}
    f(x,y) = \begin{cases}
        \frac{x|y|}{\sqrt{x^2 + y^2}} & (x,y) \neq 0\\
        0 & (x,y) = 0
    \end{cases}
\end{align*}
We can write that
\begin{align*}
    \frac{x|y|}{\sqrt{x^2 + y^2}}
    = \sqrt{x^2 + y^2}\cdot \frac{x|y|}{x^2 + y^2}
    = |(x,y)| \cdot \frac{x|y|}{|(x,y)|^2}
\end{align*}
Then defining $g$ as $g(x,y) = x|y|$ we see that
\begin{align*}
    g\bigg(\frac{(x,y)}{|(x,y)|}\bigg)
    = \frac{x}{|(x,y)|}\bigg|\frac{y}{|(x,y)|}\bigg|
    = \frac{x|y|}{|(x,y)|^2}
\end{align*}
Also, we see that $g(0,1) = g(1,0) = 0$ and that
\begin{align*}
    g(-x,-y) = -x|-y| = -xy = -x|y| = -g(x,y)   
\end{align*}
Hence we can write $f(x,y)$ as
\begin{align*}
    f(x,y) = \begin{cases}
        |(x,y)|\cdot g\bigg(\frac{(x,y)}{|(x,y)|}\bigg) & (x,y) \neq 0\\
        0 & (x,y) = 0
    \end{cases}
\end{align*}
Where $g$ is a continuous real-valued function with the same properties
described in Problem 2-4.
\\
Therefore we can conclude that $f$ is not differentiable at $(0,0)$.
\end{proof}

\cleardoublepage
\begin{proof}{\textbf{2-6.}}
Let $f: \R^2 \to \R$ be defined by $f(x,y) = \sqrt{|xy|}$, we may also define
$f$ as
\begin{align*}
    f(x,y) = \begin{cases}
        \frac{\sqrt{x^2 + y^2}\sqrt{|xy|}}{\sqrt{x^2 + y^2}} & (x,y) \neq 0\\
        0 & (x,y) = 0
    \end{cases}
\end{align*}
We see that
\begin{align*}
    \frac{\sqrt{x^2 + y^2}\sqrt{|xy|}}{\sqrt{x^2 + y^2}}
    = |(x,y)| \frac{\sqrt{|xy|}}{|(x,y)|}
\end{align*}
We can define a function $g$ as $g(x,y) = \sqrt{|xy|}$ so
\begin{align*}
    g\bigg(\frac{(x,y)}{|(x,y)|}\bigg)
    = \sqrt{\bigg|\frac{x}{|(x,y)|}\frac{y}{|(x,y)|}\bigg|}
    = \sqrt{\frac{|xy|}{|(x,y)|^2}}
    = \frac{\sqrt{|xy|}}{|(x,y)|}
\end{align*}
Also, we see that $g(0,1) = g(1,0) = 0$ and that
\begin{align*}
    g(-x,-y) = \pm\sqrt{|(-x)(-y)|} = \pm\sqrt{|xy|} = \pm g(x,y)
\end{align*}
Hence we can write $f(x,y)$ as
\begin{align*}
    f(x,y) = \begin{cases}
        |(x,y)|\cdot g\bigg(\frac{(x,y)}{|(x,y)|}\bigg) & (x,y) \neq 0\\
        0 & (x,y) = 0
    \end{cases}
\end{align*}
Where $g$ is a continuous real-valued function with the same properties
described in Problem 2-4.
\\
Therefore we can conclude that $f$ is not differentiable at $(0,0)$.
\end{proof}

\cleardoublepage
\begin{proof}{\textbf{2-7.}}
Let $f:\R^n \to \R$ be a function such that $|f(x)| \leq |x|^2$ we want to show
that $f$ is differentiable at 0.
\\
Let us consider the following limit
\begin{align*}
    \lim_{h \to 0} \frac{|f(0+h) - f(0) - Df(0)h|}{|h|}
\end{align*}
We see that
\begin{align*}
    0 \leq \frac{|f(0+h) - f(0) - Df(0)h|}{|h|}
    &\leq \frac{|f(h)| + |f(0)| + |Df(0)h|}{|h|}\\
    &\leq \frac{|h|^2 + 0 + |Df(0)||h|}{|h|}\\
    &\leq |h| + |Df(0)|
\end{align*}
We see that $|h| + |Df(0)| \to 0$ as $h \to 0$ if $Df(0) = 0$.
\\
Therefore $f$ is differnetiable at 0.
\end{proof}

\cleardoublepage
\subsection*{Basic Theorems}
\end{document}